\input paths.tex
\usetikzlibrary{graphs}
\usetikzlibrary{graphdrawing}
\usetikzlibrary{decorations}
\usetikzlibrary{decorations.pathmorphing, decorations.pathreplacing}

\usegdlibrary{force, layered}

\directlua{require("graphdrawing.lua")}

\let\oldedgecallback=\pgfgdtikzedgecallback
\def\saveedgepath#1#2{} % this macro will be defined dynamically using Lua
\def\myedgecallback#1#2#3#4#5#6#7{%
\oldedgecallback{#1}{#2}{#3}{#4, save path=\currpath}{#5}{#6}{#7}%
\passpathtolua{currpath}%
\saveedgepath{#1}{#2}}
\let\pgfgdtikzedgecallback=\myedgecallback

\newenvironment{mytikzgraphenvironment}[1][]{%
\begin{tikzpicture}[vertex/.style={circle, draw, inner sep=.2em,
                                   minimum size=1.9em},
                    annotation/.style={rectangle,inner xsep=.15em, 
                                       inner ysep=.1em, label distance=-.15cm,
                                       yshift=-.05cm,fill=white,draw},
                    node marking 1/.style={rectangle, inner sep=.1em,
                                           minimum size=1.8em},
                    node marking 2/.style={fill=darkgray, text=white},
                    node marking 3/.style={line width=1.8pt},
                    edge marking post 1/.style={line width=1.8pt},
                    edge marking post 2/.style={dashed},
                    edge marking post 3/.style={},
                    edge marking post 4/.style={decoration={name=ticks,
                                                            amplitude=1pt,
                                                            segment length=1.5pt},
                                                decorate},
                    edge marking pre 1/.style={},
                    edge marking pre 2/.style={},
                    edge marking pre 3/.style={decoration={name=snake,
                                                           amplitude=2pt,
                                                           segment length=4pt},
                                               decorate},
                    edge marking pre 4/.style={},
                    line width=.6pt, line cap=round, #1]}{\end{tikzpicture}}

\newcommand\drawgraph[2][]{%
\begin{mytikzgraphenvironment}
\directlua{g = Graph:new("#1", "#2")}%

\directlua{g:basic_layout()}
\directlua{g:draw_edges()}
\directlua{g:place_labels()}
\end{mytikzgraphenvironment}
}

\makeatletter
\newif\ifgr@phloaded
\gr@phloadedfalse
\newcommand\dg@load[2][]{\ifgr@phloaded\PackageError{graphdrawing}{You must not call \protect\load\space after using the loader environment (or a previous call of \protect\load )}{Exactly what it says above}\fi%
\gr@phloadedtrue%
\directlua{Graph:new("#1", "#2")}}
\newcommand\dg@marknode[3][]{\directlua{Graph:select_instance(#1):mark_node("#2", "#3")}}
\newcommand\dg@markedge[3][]{\directlua{Graph:select_instance(#1):mark_edge("#2", "#3")}} 
\newcommand\dg@annotate[3][]{\directlua{Graph:select_instance(#1):annotate_node("#2", "$#3$")}}
% As TeX code printed inside directlua only comes into effect once returning to
% the TeX level, we have to split these function calls into different directlua
% instances... 
\newcommand\dg@draw[1][]{\directlua{Graph:select_instance(#1):basic_layout()}%
\directlua{Graph:select_instance(#1):draw_edges()}%
\directlua{Graph:select_instance(#1):place_labels()}}

\newcommand\gl@markedge[2]{\directlua{g:mark_edge("#1", "#2")}}
\newcommand\gl@marknode[2]{\directlua{g:mark_node("#1", "#2")}}
\newcommand\gl@init[1][]{\directlua{g:update_args("#1") g:init()}}
\newcommand\gl@skip[1][1]{\directlua{g:skip(_f, #1)}}
\newcommand\gl@readnumnodes[1][]{\directlua{g:_read_num_nodes(_f, "#1")}}
\def\gl@readnumedges{\directlua{g:read_num_edges(_f)}}
\def\gl@createnodes{\directlua{g:create_nodes}}
\newcommand\gl@readmarkings[1][]{\directlua{g:_read_markings(_f, "#1")}}
\newcommand\gl@readmarking[1][]{\directlua{g:_read_marking(_f, "#1")}}
\def\gl@readannotations{\directlua{g:read_annotations(_f)}}
\def\gl@readedges{\directlua{g:read_edges(_f)}}
\def\gl@readint{\directlua{tex.print(_f:read("*number"))}}
\def\gl@annotate#1#2{\directlua{g:annotate_node("#1", "$#2$")}}

\def\beginloadgr@ph#1#2{\ifgr@phloaded\PackageError{graphdrawing}{You must not use a loader environment after a call to \protect\load\space (or another loader environment)}{Exactly what it says above}\fi%
\directlua{g = Graph:minimal_new("#1")}%
\directlua{_f = io.open("#2")}%
\begingroup%
\let\init=\gl@init%
\let\skip=\gl@skip%
\let\readnumnodes=\gl@readnumnodes%
\let\readnumedges=\gl@readnumedges%
\let\createnodes=\gl@createnodes%
\let\readmarkings=\gl@readmarkings%
\let\readmarking=\gl@readmarking%
\let\readannotations=\gl@readannotations%
\let\readedges=\gl@readedges%
\let\readint=\gl@readint%
\let\marknode=\gl@marknode%
\let\annotate=\gl@annotate%
\let\markedge=\gl@markedge%
}
\def\endloadgr@ph{\directlua{g:finish()}\endgroup\global\gr@phloadedtrue}

\newenvironment{graphpicture}[1][]{\begindrawgraph{#1}}{\enddrawgraph}
\def\begindrawgraph#1{%
\begingroup%
\global\gr@phloadedfalse%
\newenvironment{loader}[2][]{\beginloadgr@ph{##1}{##2}}{\endloadgr@ph}%
\let\load=\dg@load%
\let\marknode=\dg@marknode%
\let\markedge=\dg@markedge%
\let\drawit=\dg@draw%
\begin{mytikzgraphenvironment}%
}
\def\enddrawgraph{%
\drawit%
\end{mytikzgraphenvironment}%
\endgroup%
}
\makeatother
