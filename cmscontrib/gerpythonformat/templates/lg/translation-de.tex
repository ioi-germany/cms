\def\tTask{Aufgabe}
\def\tSubtask{Teilaufgabe}
\def\tPoint{Punkt}
\def\tPoints{Punkte}
\def\tSample{Beispiel}
\def\tSamples{Beispiele}
\def\tProgram{Dein Programm}
\def\tReturn{Rückgabewert}
\def\tExplanation{Erläuterung}
\def\tLimits{Limits}
\def\tTime{Zeit}
\def\tMemory{Speicher}
\def\tInput{Eingabe}
\def\tOutput{Ausgabe}
\def\tConstraints{Beschränkungen}


%Feedback section
\def\tFeedback{Feedback}
\def\tNofeedback{Für diese Aufgabe ist kein erweitertes Feedback verfügbar. Die angegebene öffentliche Punktzahl entspricht nur deinem Ergebnis auf den öffentlichen Testfällen und hat im Allgemeinen wenig mit deiner tatsächlichen Punktzahl gemein.}
\def\tPartialfeedback{Für diese Aufgabe ist \emph{partial feedback} verfügbar, d.h.~du bekommst das Abschneiden deines Programms auf einer festen Auswahl der Testfälle zu sehen. Die angegebene Punktzahl bezieht sich auf diese Auswahl der Testfälle und gibt damit einen Hinweis auf die endgültige Punktzahl deiner Einsendung. Letztere kann allerdings sowohl höher als auch niedriger als die angezeigte Punktzahl sein.}
\def\tFullfeedbackGeneral{Für diese Aufgabe ist \feedbackmode\ verfügbar. Das bedeutet, die angezeigte Punktzahl entspricht der endgültigen Punktzahl deiner Einsendung.}
\def\tRestrictedfeedbackPrecise{Allerdings wird dir für jede Testfallgruppe immer nur der erste Testfall mit minimaler Punktzahl innerhalb der entsprechenden Gruppe angezeigt. (Hierbei ist die Reihenfolge der Fälle innerhalb der jeweiligen Gruppen fest.)}
\def\tFullfeedbackPrecise{Ferner kannst du für jeden Testfall das jeweilige Urteil des Graders über deine Einsendung einsehen.}
\def\tTokenFeedback{Für diese Aufgabe sind \emph{Tokens} verfügbar, d.h.~du kannst für eine begrenzte Anzahl von Einsendungen vollständiges Feedback erhalten, indem du ein sogenanntes Token einsetzt. Weitere Informationen, z.B.~über die Anzahl der verfügbaren Tokens, findest du im Wettbewerbssystem.}
\def\tTokenRestrictedFeedback{Allerdings wird dir auch nach Einsetzen eines Tokens für jede Testfallgruppe immer nur der erste Testfall mit minimaler Punktzahl innerhalb der entsprechenden Gruppe angezeigt. (Hierbei ist die Reihenfolge der Fälle innerhalb der jeweiligen Gruppen fest.)}

%General overview section
\def\tOverviewSheet{Übersicht}
\def\tContestant{Teilnehmer:in}
\def\tPassword{Passwort}

%Scoring section
\def\tScoring{Bewertung}
\def\tScoringIntroduction{Je nach Aufgabe wird deine endgültige Punktzahl anders berechnet, s.~die obige Tabelle:}
\def\tScoringFromIOIXVII{Deine endgültige Punktzahl für diese Aufgabe ist die Summe über jede Teilaufgabe Deiner besten Punktzahl auf diese Teilaufgabe unter allen Deinen Einsendungen.}
\def\tScoringFromIOIXIII{Deine endgültige Punktzahl für diese Aufgabe ist die beste Punktzahl aller Deiner Einsendungen.}
\def\tScoringFromIOIX   {Deine endgültige Punktzahl für diese Aufgabe ist die beste Punktzahl aller Deiner mit Token versehenen Einsendungen und Deiner letzten Einsendung.}
\def\tScoringFromIOIXVIIgeneral{Deine endgültige Punktzahl für eine Aufgabe ist die Summe über jede Teilaufgabe Deiner besten Punktzahl auf diese Teilaufgabe unter allen Deinen Einsendungen.}
\def\tScoringFromIOIXIIIgeneral{Deine endgültige Punktzahl für eine Aufgabe ist die beste Punktzahl aller Deiner Einsendungen.}
\def\tScoringFromIOIXgeneral   {Deine endgültige Punktzahl für eine Aufgabe ist die beste Punktzahl aller Deiner mit Token versehenen Einsendungen und Deiner letzten Einsendung.}

%Task overview section
\def\tGeneralRemarks{\subsection*{Allgemeine Hinweise}
Du findest unser Wettbewerbssystem unter
\begin{center}
{\large\ttfamily contest.server-url.example}.
\end{center}
Viel Spaß und Erfolg bei der Klausur!}
\def\tTasks{Aufgaben}
\def\tName{Name}
\def\tTaskType{Aufgabentyp}
\def\tTTBatch{Batch}
\def\tTTCommunication{interaktiv}
\def\tTimeLimit{Zeitlimit}
\def\tMemoryLimit{Speicherlimit}
\def\tMaxScore{Max.~Punktzahl}
\def\tMaxScoreGeneral#1{Bei jeder Aufgabe können bis zu #1 Punkte erreicht werden.}

\def\tAND/{ und}
\def\tANDs/{ und} %The and at the end of a series of at least three items (e.g., "Eggs, Milk, and Butter."). In English, this includes the Oxford comma. In German, it doesn't.
