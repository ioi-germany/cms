\def\tTask{Task}
\def\tSubtask{Subtask}
\def\tPoint{Point}
\def\tPoints{points}
\def\tSample{Sample}
\def\tSamples{Samples}
\def\tLimits{Limits}
\def\tTime{Time}
\def\tMemory{Memory}

%Feedback section
\def\tFeedback{Feedback}
\def\tNofeedback{No additional feedback is available for this task. The score shown refers to the results on the sample case(s) and in general doesn't correspond in any way to your real score.}
\def\tPartialfeedback{Partial feedback is given for this task, i.e.~you can see the verdicts for a fixed subset of the testcases. The score shown refers to this selection of testcases and therefore \emph{hints} towards your real score, which can be either lower or higher.}
\def\tFullfeedbackGeneral{\feedbackmode\ is given for this task, i.e.~the score shown equals your real score on this submission.}
\def\tRestrictedfeedbackPrecise{However, for each testcase group you are only shown the verdict for the first testcase with minimal score in that group. (The order of the testcases in each of the groups is fixed.)}
\def\tFullfeedbackPrecise{Moreover, you are shown the verdicts for all the testcases.}
\def\tTokenFeedback{\emph{Tokens} are available for this task, i.e.~you can obtain feedback for a limited number of submissions by applying a so called token. More information, e.g.~about the number of available tokens, can be found in the contest system.}
\def\tTokenRestrictedFeedback{However, even when you've applied a token, for each testcase group you are only shown the verdict for the first testcase with minimal score in that group. (The order of the testcases in each of the groups is fixed.)}

%General overview section
\def\tOverviewSheet{Overview Sheet}
\def\tContestant{Contestant}
\def\tPassword{Password}

%Scoring section
\def\tScoring{Scoring}
\def\tScoringIntroduction{The way your final score is computed depends on the given task as indicated in the above table:}
\def\tScoringFromIOIXVII{Your final score for this task is the sum over each subtask of your best score on that subtask across all your submissions.}
\def\tScoringFromIOIXIII{Your final score for this task is the best score of all your submissions.}
\def\tScoringFromIOIX   {Your final score for this task is the best score of all your tokened submissions and your last submission.}
\def\tScoringFromIOIXVIIgeneral{Your final score for a given task is the sum over each subtask of your best score on that subtask across all your submissions.}
\def\tScoringFromIOIXIIIgeneral{Your final score for a given task is the best score of all your submissions.}
\def\tScoringFromIOIXgeneral   {Your final score for a given task is the best score of all your tokened submissions and your last submission.}

%Task overview section
\def\tGeneralRemarks{\subsection*{General remarks}
You can find our contest server at
\begin{center}
{\large\ttfamily contest.server-url.example}.
\end{center}
Good luck and have fun!}
\def\tTasks{Tasks}
\def\tName{Name}
\def\tTaskType{Type}
\def\tTTBatch{batch}
\def\tTTCommunication{communication}
\def\tTimeLimit{Time limit}
\def\tMemoryLimit{Memory limit}
\def\tMaxScore{Max.~score}
\def\tMaxScoreGeneral#1{You can obtain up to #1 points for each task.}

\def\tAND/{ and}
\def\tANDs/{, and} %The and at the end of a series of at least three items (e.g., "Eggs, Milk, and Butter."). In English, this includes the Oxford comma.
